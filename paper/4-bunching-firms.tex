\section{Firm Non-standard Behavior and Wage Bunching} \label{sec:firm-behavior}

This section investigates whether the bunching observed in the data is driven by non-standard behavior of workers or firms. The approach taken here involves studying the characteristics and outcomes of firms that tend to hire workers at round numbers and assessing whether these are consistent with the hypothesis that the firms that pay round numbers do so to exploit a worker bias. 

\subsection{Defining Bunching Firms} 

I start by measuring a firm's propensity to hire workers at round-numbered salaries. As a simple and intuitive measure, I compute the fraction of a firm's new hires over 2003--2017 whose initial salary is a round number.\footnote{I exclude from this computation workers hired at or below the minimum wage and those without a monthly earnings contract.} I focus on hiring at salaries divisible by 10 for consistency with previous research on the clustering of wages at round numbers \citep[e.g.,][]{riddles_handling_2016}, and show that the results below are robust to defining bunching firms using coarser salaries (e.g., those divisible by 100).

Round-number wage-setting is highly heterogeneous across firms, with many firms only hiring workers at round-numbered salaries (Appendix Figure \ref{fig_hist_bunch_firms}). In the data, one in six firms (16.9\%) exclusively hired workers at round salaries. I refer to these as \textit{bunching firms}. Appendix Table \ref{reg_firm_char} compares the characteristics of bunching and non-bunching firms.

The fraction of bunching firms is higher under less stringent definitions. For instance, 33.2\% [27.1\%] of firms hired more than half [two-thirds] of their new workers at a round salary. In Appendix Tables \ref{reg_firm_performance_size}--\ref{reg_salary_increase_def}, I show that the results below are robust to these alternative definitions of bunching firms. The results are also robust to excluding small firms (i.e., firms that employ fewer than five workers) and using the \textit{yearly} salary of new hires to define bunching firms (instead of the monthly salary). 

\subsection{The Market Outcomes of Bunching Firms} \label{sub:outcomes-bunchers}

A potential rationale for why many firms pay round-numbered salaries is to extract surplus from workers who have non-standard preferences. If this hypothesis holds true, one should see the consequences reflected in better firm outcomes. To evaluate this, I consider four outcomes: worker separation and resignation likelihoods during the hiring year or the following year, which I use as proxies of a poor worker-firm match; the growth rate of the firm's size, as measured by its number of employees; and a binary variable indicating if a firm leaves the market. While I do not observe firm profit, the firm growth and survival rates are functions of realized profits. 
 
In Table \ref{reg_firm_performance}, I estimate regressions of the form:
%
\begin{align} \label{reg:firm-outcomes}
	y_{ijt} = \alpha + \beta \text{BunchingFirm}_{j} + \psi X_{it} + \delta Z_{jt} + \varepsilon_{ijt},
\end{align}
% 
where $i$ denotes workers, $j$ firms, and $t$ years; $y_{ijt}$ is one of the four outcomes; and $\text{BunchingFirm}_{j}$ equals one if firm $j$ hired all new employees at a round salary in the sample. 

The regression includes $X_{it}$, a vector of worker characteristics (age, gender, race, and occupation dummies), and $Z_{jt}$, a vector of fixed and time-varying firm characteristics that are typically associated with firm sophistication (see Appendix \ref{app:var-def} for variable definitions). These characteristics are the presence of a human resources (HR) department, the share of employees with a high-school and college degree, educational attainment of the manager, firm age, mean earnings of the firm employees, firm size, and firm hiring experience. I flexibly control for firm size and hiring experience by including fixed effects for the number of workers hired and the mean number of workers employed. $Z_{jt}$ also includes industry-by-year-by-microregion fixed effects.\footnote{A microregion is a geographical area that groups together economically integrated contiguous municipalities with similar productive structures. There are about 500 microregions in Brazil, each of which can be thought of as a local labor market (see \citealp{dix2017trade}). The boundaries of these microregions are defined by the National Statistics Office of Brazil.}

To analyze worker separation likelihood, I estimate the regressions at the worker-by-firm-by-year level. To analyze the firm growth and survival rates, I estimate the regressions at the firm-by-year level (and exclude the worker controls). I cluster the standard errors at the firm level.

Table \ref{reg_firm_performance} shows that bunching firms tend to have worse outcomes. Columns 1 and 2 show that new hires in bunching firms are, on average, 4.1 percentage points (an 11.6\% increase relative to the sample mean) and 1.2 percentage points (or 10.4\%) more likely to separate and resign, respectively, than new hires in non-bunching firms ($p<0.01$). Column 3 shows that bunching firms have a 3.4 percentage points lower growth rate, on average, than non-bunching firms ($p<0.01$). Column 4 shows that bunching firms are 1.1 percentage points (or 10.8\%) more likely to exit the market ($p<0.01$). 

These results are robust to excluding small firms, varying the set of controls, and using alternative definitions of bunching firms. Appendix Table \ref{reg_firm_performance_size} estimates the baseline specification for the subset of firms that employ more than five workers, on average across all years. The same results hold for these large firms. Appendix Table \ref{reg_firm_performance_lev} estimates all specifications at the worker level and additionally controls for the wage level by including wage fixed effects (in R\$100 bins). The point estimates are quantitatively similar to those of the baseline specification. Finally, Appendix Table \ref{reg_firm_performance_def} shows that the results are robust to defining bunching firms in several alternative ways.

In Appendix \ref{app:outcomes}, I study additional outcomes. Specifically, I analyze whether the higher separation rates of new workers hired by bunching firms persist over time, whether the higher separation rates are driven by high- or low-skilled workers, and assess whether the lower job growth rate of bunching firm is driven by high- or low-skilled workers. I find that the higher separation rates persist up to three years after new hires join bunching firms (Appendix Table \ref{reg_firm_performance2}). The higher separation and resignation rates of new workers hired by bunching firms are mainly driven by high-skilled workers (Appendix Table \ref{reg_firm_performance3}). Furthermore, bunching firms experience lower job growth rates for both high-skilled and low-skilled workers, as well as for high- and low-paid employees (Appendix Table \ref{reg_firm_performance4}).

These results admit several interpretations. First, the worse outcomes experienced by bunching firms may partly be a consequence of paying round-numbered salaries. For example, the higher separation likelihood might be due to a poor worker-firm match caused by paying a non-optimal wage. In addition, the results can also be explained by bunching firms being less sophisticated in other unobserved dimensions, which in turn might drive their worse outcomes. For example, in addition to having non-standard pay-setting practices, bunching firms may have less efficient production processes, which may be the cause of the lower survival rates. Finally, the round-number wage-setting may be the results of the constrained optimization problem faced by firms. For example, managers might (optimally) prioritize spending resources to improve production efficiency over pay-setting strategies. Regardless of the right explanation, the results at odds with the hypothesis that sophisticated firms pay round-numbered salaries to exploit a worker bias.

\subsection{Behavior of Bunching Firms in a Different Decision Environment}  \label{sub:heuristics-bunchers}

Another possible explanation for why many firms pay round-numbered salaries is that they are uncertain about what the fully-optimal salary is and use round-numbered salaries as a simple but coarse approximation.\footnote{Simplified pricing strategies have been documented in several environments \citep[e.g.][]{cho2010flat, dellavigna2019uniform, stevens2020coarse}.} For example, firms might be uncertain about worker productivity, which is a central determinant of the optimal salary in wage-determination models.

If this hypothesis is true, one would also expect these firms to rely on similar coarse approximations in other environments where they also face uncertainty. To evaluate this, I use salary increases as a different domain to explore the potential use of coarse pay-setting. In this domain, firms face uncertainty about employee realized productivity. The canonical Bayesian model of wage formation predicts that a worker's wage increase depends on her realized productivity \citep{jovanovic1979job, tervio2009superstars}. By contrast, coarse pricing suggests that firms determine raises based on coarse approximations, such as integer numbers if the salary increase is measured in percentage terms or round numbers if the increase is measured in monetary units.

Table \ref{reg_salary_increase} shows estimates of equation \eqref{reg:firm-outcomes} using two measures of coarse pay-setting as outcomes: first, a dummy that equals one if a new hire received a round-numbered salary increase in Brazilian Reals (e.g., R\$310 as opposed to R\$314), and second, a dummy that equals one if a new hire received an integer salary increase in percentage terms (e.g., 3\% as opposed to 3.14\%).

Firms that tend to hire workers at round salaries also tend to rely on coarse figures when deciding salary increases. Columns 1 and 3 show that bunching firms are 29.3 percentage points more likely to offer a round-numbered salary increase in Brazilian Reals (from a baseline of 35.4\%, $p<0.01$) and 16.2 percentage points more likely to offer an integer salary increase in percent terms (from a baseline of 34.4\%, $p<0.01$). Column 5 shows that bunching firms are about 28.6 percentage points more likely to engage in either of the two behaviors (from a baseline of 42.1\%, $p<0.01$). 

These results are robust to excluding workers whose salaries remained constant in nominal terms (columns 2, 4, and 6). They are also robust to excluding small firms (Appendix Table \ref{reg_salary_increase_size}), controlling for the wage level (Appendix Table \ref{reg_salary_increase_lev}), and defining bunching firms in several alternative ways (Appendix Table \ref{reg_salary_increase_def}).
