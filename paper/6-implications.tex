\section{Implications for Other Economic Outcomes} \label{sec:implic}

In this section, I explore some of the downstream consequences of firm coarse wage-setting for important economic outcomes.

\subsection{Within-Firm Wage Inequality}

Understanding the drivers of wage inequality is an important research agenda in public and labor economics. Previous research has found that firm wage-setting policies influence wage inequality \citep{card2018firms}. One might expect firm coarse wage-setting to affect wage dispersion among new hires.\footnote{Ex-ante, the direction is ambiguous. To see this, consider a firm that pays workers their fully-optimal salary rounded to the nearest 1,000th deciding the wages of two new hires. If the workers' fully-optimal salaries are R\$700 and R\$1,400, but the firm pays both of them R\$1,000, then the coarse pricing generates wage compression. Instead, if the first worker's fully-optimal salary is R\$1,700, the paid salaries would be R\$2,000 and R\$1,000, respectively. In this case, the coarse wage-setting increases wage dispersion.} To assess this, I estimate equation \eqref{reg:firm-outcomes} using as outcomes the Gini coefficient and ratios between the contracted salary at the 90th and 10th percentile, 90th and 50th percentile, and 50th and 10th percentile.\footnote{The Gini measures overall inequality in the contracted salary distribution, while the ratios measure inequality at different parts of the distribution (e.g., top- or low-end inequality, see \citealp{lemieux2008changing}).} Since equation \eqref{reg:firm-outcomes} includes fixed effects for the number of workers hired, the research design compares the within-firm wage inequality of two firms that hired the same number of workers using a different decision rule to determine their initial pay.

Figure \ref{fig_wage_comp} shows that bunching firms tend to compress wage differentials among new hires (see Appendix Table \ref{reg_wage_compression} for the corresponding regression coefficients). The average Gini coefficient among non-bunching firms is 0.112 (Panel A).\footnote{By country standards, this is a very low level of inequality. The most egalitarian countries in the world---typically, the Nordic countries---have a Gini coefficient on the order of 0.25. Two reasons explain the difference in magnitudes. First, country-level inequality is typically measured using household \textit{consumption} per capita as the welfare measure, whereas I compute the Gini using earnings. Second, I calculate the Gini among new hires of a given firm, which is likely a more homogeneous population than the overall population of a country.} Bunching firms have a 0.01 lower Gini coefficient (or 8.9\% of the baseline value). The decline in overall wage inequality is driven by mostly top- and mid-end inequality (Panel B). The ratio between the 90th and 10th percentile is, on average, 3.9\% lower (from a baseline ratio of 1.76) in bunching firms relative to the rest of the firms. Similarly, bunching firms have, on average, a 2.8\% lower 90th to 50th percentile ratio and a 1.6\% lower 50th to 10th percentile ratio than non-bunching firms (from baseline ratios of 1.39 and 1.24, respectively). These effects are robust to excluding small firms (Appendix Figure \ref{fig_wage_comp_bigf}). 

\subsection{Nominal Wage Rigidity}

Nominal wage stickiness influences the effects of monetary policy \citep{barattieri2014some}. Previous work has documented that behavioral considerations such as inertia \citep{eichenbaum2011reference}, managerial inattention \citep{ellison2018costs}, and fairness norms \citep{kaur2019nominal} influence nominal rigidities. Coarse wage-setting might contribute to wage rigidity if it makes firms less likely to change the initial salary of their new hires. To assess this, I estimate equation \eqref{reg:firm-outcomes} using as the dependent variable a dummy that equals one if the nominal salary of a new hire remained constant in nominal terms during the year following the hiring, and zero otherwise. 

The initial salaries of bunching firms' workers tend to be stickier (Figure \ref{fig_wage_comp}, Panel C). From a baseline of 26.0\%, workers employed by bunching firms have a 13-percentage-point increase in the probability of experiencing no salary change. Thus, relative to new hires of non-bunching firms, those employed by bunching firms are about 50\% more likely to exhibit nominal wage stickiness. This effect is robust to excluding small firms (Appendix Figure \ref{fig_wage_comp_bigf}).

\subsection{Minimum Wage Spillovers}

\cite{dube_monopsony_2020} hypothesize that in the presence of firms that pay round-numbered wages, a change in the minimum wage could generate a novel spillover effect if the new minimum wage crosses a round number. Intuitively, a change in the minimum wage might cause firms that initially pay a round-numbered wage to fully optimize. However, their data does not allow them to test this hypothesis. In my sample, I observe hiring decisions under fifteen different federal minimum salaries, seven of which are round numbers. I also observe the year $t+1$ salary of workers hired in year $t$, which allows me to assess the importance of this potential spillover effect. 

I describe the methodology and results in detail in Appendix \ref{app:min-wage}. In short, using a differences-in-differences approach comparing salaries directly affected by the change in the minimum salary and those not directly affected by it, I find that an increase in the minimum salary reduces the share of round-numbered salaries by 5.4 percentage points (or 11.3\%). This finding suggests that changes in the minimum wage can have sizable spillover effects on firm wage-optimization behavior.
