\section{Bunching in the Distribution of Contracted Salaries}  \label{sec:anomalies}

The two main classes of wage-formation models in labor economics are wage-posting models and wage-bargaining models \citep{manning_imperfect_2011}. Under standard assumptions, both types of models predict a smooth distribution of wages at the market level (see Appendix \ref{app:labor-models}). 

The data unequivocally rejects this prediction. Figure \ref{fig_bunching}, Panel A plots the distribution of contracted salaries in the new-hires sample. The earnings distribution exhibits stark bunching at round numbers (i.e., numbers divisible by 10). For example, workers are fifteen times more likely to earn exactly R\$3,000 per month than any other salary between R\$3,001 and R\$3,010. The modal monthly salary in the new-hires sample is R\$1,000, followed by R\$800, and R\$600 (jointly accounting for over six million contracts), with all being round numbers.

The bunching is also manifested in a non-uniform distribution of the last digit of salaries. Figure \ref{fig_bunching}, Panel B shows the fraction of salaries that are divisible by 10, 100, and 1,000. About one-third of the salaries (29.5\%) in the new-hires sample are divisible by 10 (see also Appendix Figure \ref{fig_last_digits}). This figure would be 10\% if the last digits of salaries were uniformly distributed. Over one-tenth of salaries (12.1\%) are divisible by 100 (a uniform distribution would imply 1\%), and 1.9\% are divisible by 1,000 (a uniform distribution would imply 0.1\%). 

These figures likely underestimate the true degree of bunching for several reasons. First, the contracted salary might be a round number at a different periodicity. For instance, over one-tenth (10.7\%) of the salaries that are not round numbers at the monthly level are round numbers at the yearly level. Similarly, some non-round-numbered salaries might be due to firms setting the wages of new hires equal to the wage of current employees. Some of these wages might have started initially as a round number but were updated over time into non-round-numbered wages.

\subsection{Bunching of Salaries at Round Numbers in Four Other Datasets}

To provide further evidence on the existence and magnitude of bunching, I study the earnings distribution in four additional datasets: the 2013 Brazilian Household Survey (\textit{Pesquisa Nacional por Amostra de Domicílios}, PNAD), the 2013 Brazilian Labor Force Survey (\textit{Pesquisa Mensal de Emprego}, PME), the 2010 Brazilian Population Census (\textit{Censo Demográfico}) and the 2013 Social Programs Registry of Individuals (\textit{Cadastro Único}).\footnote{The PNAD is a nationally-representative survey conducted annually by the National Statistics Office to measure several characteristics of the population, such as household composition, education, and income. The PME is a monthly survey conducted in six large metropolitan areas to provide frequent updates on the unemployment rate and other labor-market variables. The census is conducted approximately every ten years to count the population in the country, but it also includes earnings information. Finally, the Social Programs Registry contains information on all beneficiaries of government programs, including their earnings.} In all datasets, I focus on the monthly salary of full-time workers aged 18--65. I exclude workers employed by public-sector firms and individuals who work without remuneration. 

The advantage of these datasets is that they include information on informal-sector workers, while the main disadvantage is that earnings are self-reported. Hence, worker salaries might be measured with error due to---for example---recollection bias or social-desirability bias. Another drawback is that the labor income measure refers to the earnings during the month before the survey was conducted and not the contracted earnings when the employer hired the worker. 

Figure \ref{fig_rounding_datasets} shows the fraction of monthly earnings divisible by 10, 100, and 1,000 in each dataset (see Appendix Figure \ref{fig_bunching_data} for the  earnings distribution). All datasets exhibit stark bunching at round numbers. For example, 96.1\% of monthly earnings in the census are divisible by 10. The corresponding figure in the Household Survey is 94.1\%, in the Labor Force Survey it is 96.5\%, and in the Social Programs Registry it is 79.2\%. This provides additional evidence against the hypothesis that salaries are smoothly distributed. The fact that we do not observe such an extreme bunching in the RAIS is consistent with previous research showing that the bunching in surveys is partly driven by recollection bias from the respondent side, although it could also reflect informal-sector firms paying round-numbered salaries at a higher rate.

Taken together, the results of this section show that bunching at round numbers is a ubiquitous feature of labor markets and not simply a consequence of measurement error.