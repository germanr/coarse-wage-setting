\section{Discussion} \label{sec:conclusions}

Setting the right wage is challenging. To estimate the fully-optimal wage prescribed by economic models, a firm needs substantial information, including an estimate of the worker's contribution to the firm. Most workers have multiple goals and no measured output, which makes productivity difficult to estimate. This paper posits that the stark bunching at round numbers in the earnings distribution partly reflects the challenges associated with optimal labor pricing. In the data, millions of workers are hired at round-numbered salaries, reflecting a behavior that cannot be accommodated by existing wage-setting models. The evidence presented in this paper indicates that this behavior is partly due to firms engaging in coarse wage-setting. 

An important unresolved question is whether the coarse wage-setting is suboptimal. Setting optimal pay-setting practices likely requires substantial resources. If these costs are large, offering a coarse wage might lead to better outcomes. Nonetheless, the findings have intrinsic value for understanding how firms set wages. Coarse wage-setting may also ahve consequences for wage inequality, nominal wage rigidity, and may interact with policies that affect the wage distribution.

Future work could also explore the extent to which rounding reflects the quality of management practices. Management quality is often not available in traditional datasets (the World Management Survey is a notable exception, see \citealp{bloom2007measuring}). If coarse pricing partly reflects how human resources are managed at the firm, researchers could use the type of salaries offered to new hires as a proxy for overall HR management quality.
