\section{Institutional Context, Data, and Descriptive Statistics} \label{sec:context-data}

This section provides institutional context on Brazil's labor market, describes the administrative dataset, and provides descriptive statistics of the samples.

\subsection{Brazil's Labor Market} \label{subsec_context}

Brazil's labor market has both a formal and an informal sector (see Appendix \ref{app:inform}). I focus on the formal sector, which employs about 80\% of wage employees and has a strict labor code. The contracts of formal-sector workers are governed by the Brazilian Labor Code, which mandates provisions such as a relatively high minimum wage, an extra monthly salary annually, one month of paid leave each year, and high firing costs.

\subsection{Data: Employee-Employer Matched Information} \label{sub:data}

The main data source is the \textit{Relação Anual de Informações Sociais} (RAIS), an employee-employer matched dataset covering the universe of formal-sector jobs in Brazil from 2003 to 2017. This administrative dataset is assembled yearly by the Ministry of Labor with information provided by firms. Accurate reporting in the RAIS is required for workers to receive payments from some government programs, and firms face financial penalties for not reporting.\footnote{The main drawback of the RAIS is that it only contains information on formal-sector workers and firms. Thus, the analysis is not representative of the informal sector. Appendix \ref{app:inform} uses data from the Brazilian Household Hurvey, which includes data on informal-sector workers, to compare workers in the RAIS to workers in the entire labor force.}

The RAIS contains both firm- and worker-level information. Firms' data include the number of employees, industry, and location. Workers' data include educational attainment, occupation, and employment information, including the hiring date, recruitment type (e.g., new hire, transfer, etc.), and contracted salary.

The contracted salary of a worker is central to the empirical analysis. It is the salary contained in an individual's ``Work and Social Security Card'' (or CTPS) at the end of each year.\footnote{Appendix \ref{app:handbook} provides an example of a CTPS and the information it contains.} The CTPS documents a worker's employment history, including the initial salary at the firm and any subsequent modifications. For a new hire, the contracted salary is the initial salary at which the firm hired the worker. For other workers, the contracted salary might differ from the initial salary due to a raise or promotion, for instance.

\subsection{Samples and Descriptive Statistics} \label{sub:summ}

\textit{New-hires sample.} For much of the empirical analysis, I use a \textit{new-hires sample}, in which each observation represents a new hire (defined by a worker-firm-hiring date triplet). To construct this sample, I only include new workers hired each year by private-sector firms. For workers holding multiple positions at a firm in a given year, I only keep the highest-paying position. I exclude new hires without a valid identification number or a reported contracted salary below the federal monthly minimum wage. Finally, I only keep new hires with a monthly earnings contract. This excludes, for example, workers who bill by the hour or per day worked, which constitute a small fraction of workers in the data. After imposing these restrictions, the database contains information on the contracted salary of 206 million hires (henceforth, ``contracts'' or ``workers'' for short) from 2003 to 2017.\footnote{Appendix \ref{app:samp-rest} provides more detail on each of these steps and show the fraction of excluded observations after each sample restriction.}

\textit{Firm random sample.} To conduct any analysis that requires exploiting the panel structure of the dataset, I select a random sample of firms. To construct this sample, I create a registry of all private-sector firms ever observed in the RAIS who hired at least one worker at a monthly earnings contract during 2003--2017. Due to computational constraints, I randomly select 5\% of them. I track all of the employees of these firms over time (both new hires and other employees). This sample includes over 300,000 firms, 1.8 million firm-years, and 31.8 million worker-years.

\textit{Descriptive statistics.} Table \ref{tab_rais_summ} presents summary statistics of workers in the RAIS and the samples. The average worker in the new-hires sample is 30.3 years old. Most workers are male (63.5\%), white (57.4\%), and have completed high school (53.2\%). The average monthly salary is R\$807 (approximately \$373). Most workers are employed by the retail industry (35.9\%), followed by the services industry (26.5\%). Workers in the firm random sample have similar characteristics.
