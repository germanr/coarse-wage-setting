\section{Introduction}

The wage-formation process is a central element of many economic models. While standard wage-formation models assume that workers and firms behave optimally, recent findings challenge this assumption. Both survey and administrative data reveal a tendency for wages to cluster---or ``bunch''---at round numbers \citep{riddles_handling_2016, dube_monopsony_2020}. This puzzling finding suggests non-standard behavior by some market participants. According to one view, the bunching is driven by strategic behavior on the firm side, whereby firms pay round-numbered wages to exploit a worker behavioral bias (e.g., left-digit bias). Alternatively, the bunching might reflect the behavior of firms engaging in non-standard wage-setting, possibly due to misoptimization.\footnote{Throughout the paper, I use the term ``non-standard'' to refer to any behavior that departs from the predictions of the neoclassical model. Specifically, ``non-standard wage-setting'' refers to firm wage-setting practices that depart from the first-order condition of canonical wage-formation models. See Appendix \ref{app:theory} for a description of the two main classes of wage-setting models.}

In this paper, I use rich worker-firm matched data to assess whether the wage bunching is partly due to firm non-standard wage-setting. First, I establish the existence of substantial bunching at round-numbered salaries in the data. Then, I provide a set of reduced-form results compatible with firms engaging in coarse wage-setting and inconsistent with firms paying round-numbered wages to exploit a worker bias. Motivated by the reduced-form findings, I develop a wage-posting model in which firms pay coarse rounded salaries due to optimization costs. The model delivers two predictions for which I find support using two research designs. Finally, I quantify some of the downstream consequences of coarse wage-setting for within-firm wage inequality, nominal wage stickiness, and policies that affect the wage distribution, such as changes in the minimum wage.

I use an administrative employee-employer matched dataset covering the universe of formal-sector firms in Brazil from 2003 to 2017. This dataset contains the salary at which firms hire workers (``contracted salary''). I use data on the contracted monthly salary of over 200 million new hires. In addition, I use a sample of over 300,000 firms that includes information on all of their employees. Using this data, I document the existence of substantial bunching at round-numbered salaries (i.e., those divisible by 10) in the distribution of contracted salaries. For example, 33.8\% of new hires' contracted salaries are round numbers (a uniform distribution would imply 10\%). My findings stand in opposition to the predictions of canonical wage-determination models, in which market-level wages should be smoothly distributed. 

Then, I present a series of reduced-form results to shed light on whether worker or firm non-standard behavior drives the bunching. I begin by identifying a set of firms (``bunching firms'') that tend to hire workers at round-numbered salaries. Next, I compare the market outcomes of bunching firms with those of non-bunching firms. If bunching firms pay round-numbered wages to exploit a worker bias, one would expect these firms to have better market performance than non-bunching firms. However, I find that conditional on a large set of controls, bunching firms tend to experience worse outcomes. For instance, they have worse worker-firm matches as measured by new hires' separation likelihood, a lower job growth rate, and they are more likely to exit the market. 

These reduced-form results suggest that bunching firms do not pay round salaries to exploit a worker bias. Alternatively, firms may pay round-numbered salaries as a simple but coarse approximation when they are uncertain about what the fully-optimal salary is. When hiring a new worker, firms face considerable uncertainty about a worker's marginal revenue product (or ``productivity''). Estimating a worker's contribution to the firm requires answering complex questions: What are all of the possible tasks that the new hire will perform? How does each of these tasks affect the firm's bottom line? How likely is the prospective employee to successfully accomplish each of these tasks? Instead of attempting to gather all of the information required to compute worker productivity, firms might rely on a rule-of-thumb or heuristic as an approximation---a form of pricing that I refer to as ``coarse wage-setting.''

As a suggestive reduced-form test for the use of coarse wage-setting, I assess whether bunching firms also rely on coarse figures when deciding on salary increases. This is a different environment in which firms also face uncertainty about the optimal action. The canonical model predicts that workers' wage increase depends on their realized productivity \citep{jovanovic1979job}. Since this variable is difficult to measure, some firms might use coarse approximations as salary increases, such as integer numbers if the salary increase is measured in percentage terms or round numbers if the increase is measured in monetary units. I find that bunching firms also rely on coarse approximations while deciding wage increases. Bunching firms are 26 percentage points more likely to offer a round-numbered salary increase in monetary units (from a baseline of 20.4\%) and 9 percentage points more likely to offer an integer salary increase in percent terms (from a baseline of 12.9\%).

The reduced-form results motivate the hypothesis that coarse wage-setting is partly what drives the bunching observed in the data. To further explore this hypothesis, I build a wage-posting model in which coarse wage-setting is a consequence of optimization costs. The goal of the model is to account for the bunching observed in the data and to generate ancillary predictions that should hold if firms engage in coarse wage-setting. The model relies on two key assumptions that I motivate based on numerical cognition research: first, that firms use a rounding heuristic to form an initial estimate of the fully-optimal salary (the salary firms would pay if there were no optimization costs), and second, that firms can generate a more precise estimate of the fully-optimal salary by paying some cost. The standard wage-posting model is a special case of the model with frictions, in which the optimization cost is zero.

The model delivers two testable predictions that characterize the conditions under which firms are more likely to hire workers at coarse round-numbered wages. First, a smaller expected gap between the coarse wage and the fully-optimal wage should increase the likelihood of firms paying the coarse wage. In the model, the firm's benefit of fully optimizing is proportional to this gap. Hence, as this benefit decreases, firms are less likely to pay the cost of computing the fully-optimal wage. Second, firms with lower optimization costs should be less likely to pay coarse wages and more likely to pay the fully-optimal wage. 

I test the model's predictions using two research designs. The first empirical strategy is a bunching design that uses standard techniques in the bunching literature \citep{kleven_bunching_2016}. This design consists of correlating the fraction of workers hired through a coarse rounding heuristic with firm and worker characteristics. I partition the data based on firm and worker characteristics and recover the fraction of workers hired through a coarse wage-setting using the ``excess mass'' in the density of workers earning round-numbered salaries. Second, I estimate linear probability models with firm fixed effects, where the dependent variable is an indicator for paying a round-numbered salary to a new hire. This design allows me to control for a large set of confounding factors, including unobserved heterogeneity at the firm level. The two research designs deliver similar results in support of the model's predictions. 

% This provides additional evidence consistent with the firm coarse wage-setting hypothesis. 

While my findings are consistent with non-standard firm wage-setting, they do not imply that firms are making mistakes. For instance, firms may be adopting management practices that, despite leading to coarse wage-setting, may improve overall firm performance. Still, these findings are important for three main reasons. %
%
First, round-numbered wages make up a disproportionate share of all wages. Thus, understanding why firms pay round-numbered wages sheds light on overall firm wage-setting and can inform the modeling assumptions of wage-setting models. % 
%
Second, research designs that infer parameter values from firm optimality conditions might yield biased estimates due to coarse wage-setting. Researchers within the structural tradition often infer unobservable variables using the firm's first-order conditions (FOC).\footnote{For example, in the context of a wage-posting model, a researcher equipped with wage data and an estimate of worker productivity could use a firm's FOC to identify the labor supply elasticity. This particular strategy has gained traction in recent years as researchers are increasingly interested in understanding imperfect competition in the labor market \citep[e.g.,][]{lamadon2022imperfect}.} However, if firms do not fully optimize with respect to wages, the FOC may not characterize firms' pricing decisions. % 
%
Third, coarse wage-setting can have downstream consequences for important economic outcomes. I show that coarse wage-setting may lead to within-firm wage compression and may increase wage stickiness. In addition, in the presence of firms that engage in coarse wage-setting, policies that affect the earnings distribution can affect firm wage-optimization behavior.

This paper contributes to empirical studies of firm wage-setting.\footnote{See, among others, \cite{hall2012evidence,caldwell2019outside, hjort2020across, derenoncourt2021spillover, lachowska2021wage, cullen2022s, hazell2022national}.} At least since \cite{jones1896round}, labor economists have documented the bunching of salaries at round numbers. Some work has suggested that this bunching arises from measurement error \citep[e.g.,][]{schweitzer_rounding_1996}. I contribute by documenting bunching in an administrative dataset where earnings are not self-reported, which shows that round-number bunching is a real feature of labor markets.\footnote{Other papers have also documented the bunching of earnings at round numbers \citep[e.g.,][]{kleven2013using, devereux2014elasticity, mavrokonstantis2022bunching}. In these papers, the excess mass of salaries at round numbers is a nuance parameter, not the object of interest.} The most closely related paper is \cite{dube_monopsony_2020}. Using unemployment insurance records from the US, they document substantial bunching at \$10 per hour and show that this pattern is not explained by worker left-digit bias. This raises the possibility that firm non-standard behavior drives the bunching. I contribute in two main ways, first by establishing a novel set of stylized facts about firms that frequently hire workers at round-numbered wages, and second by quantifying some downstream consequences of such coarse wage-setting on important outcomes.

This paper also contributes to an emerging empirical literature on simplified firm pricing strategies. The view that firms set prices based on heuristics and simplified rules dates back to \cite{simon1962new}, who noted that ``price setting involves an enormous burden of information gathering and computation that precludes the use of any but simple rules of thumb as guiding principles.'' Recent empirical work provides evidence to support Simon's claim. For example, \cite{cho2010flat} show that car companies charge a uniform rental price across cars with heterogeneous odometer values, \cite{cavallo2014currency} find that global retailers engage in uniform pricing across heterogeneous countries, and \cite{dellavigna2019uniform} show that US retail chains engage in uniform pricing across heterogeneous outlets.\footnote{Other work shows that many firms follow coarse pricing policies \citep[e.g.,][]{matejka_rationally_2016, stevens2020coarse}.} While these papers focus on the goods market, I contribute by documenting a form of simplified pricing in the labor market.

Finally, this paper contributes to the literature studying market outcomes in the presence of behavioral firms. Compared to the ever-growing number of papers that document biases in individuals' behavior, work on firm heuristics and biases is scarce.\footnote{See \cite{heidhues_behavioral_2018} for a theoretical overview of this literature and \cite{kremer2019behavioral} for work on behavioral firms in developing countries. Among the empirical papers that study behavioral firms, previous work has shown that entrepreneurs are overconfident regarding future growth \citep{landier2008financial}, restaurant owners do not account for the transitory nature of weather shocks \citep{goldfarb2019transitory}, car dealerships exhibit loss aversion \citep{pierce2020negative}, and retailers underestimate the degree of consumers' left-digit bias \citep{strulov2019more}. A closely related body of work documents firms' failure to maximize profits \citep[e.g.,][]{hanna2014learning, bloom2013does, almunia2021strategic}.} This is partly due to data limitations. Most of the heuristics and biases body of work studies individuals' behavior in carefully-controlled lab environments. There is not a straightforward way of conducting the same type of experiments using firms as research subjects. I contribute by providing field evidence on firm non-standard behavior in a high-stakes setting. 
